\documentclass[10pt]{beamer}
\usepackage[utf8]{inputenc}
\usepackage{polski}
\author{Marcin TORGiren Fabrykowski}
\title{No i Git!\\część I}
\institute{AGH - University of Science and Technology}
\usetheme{Warsaw}
\begin{document}
\begin{frame}
	\titlepage
\end{frame}
\section{Trochę teorii}
	\subsection{Definicja}
		\begin{frame}
			\frametitle{System kontroli wersji}
			\begin{block}<1->
				{główne zadanie}
				kontrola i~zachowywanie wszystkich wersji kontrolowanych plików
			\end{block}
			\begin{block}<2->
				{zadania dodatkowe}
				\begin{itemize}
					\item<3-> wielu użytkowników
					\item<4-> rozwijanie w~różnych kierunkach
					\item<5-> backup
				\end{itemize}
			\end{block}
		\end{frame}
	\subsection{Podział}
		\begin{frame}
			\frametitle{Podział}	
			\begin{block}<1->
			{lokacja}
			\begin{itemize}
				\item<2-> scentralizowane
				\item<3-> rozproszone
			\end{itemize}
			\end{block}
			\begin{block}<4->
			{przechowywanie wersji}
			\begin{itemize}
				\item<5-> jako zmiany
				\item<6-> jako snapshoty
			\end{itemize}
			\end{block}
		\end{frame}
\section{Podstawy}
\subsection{Pierwsze kroki}
\begin{frame}
	\frametitle{Pierwszy commit}	
	\begin{block}<1->
	{klonujemy repozytorium}
		git clone git@192.168.246.133:first.git
	\end{block}
	\begin{block}<2->
	{praca nad projektem}
	vim plik 
	\end{block}
	\begin{block}<3->
	{dodanie pliku}
	git add plik
	\end{block}
	\begin{block}<4->
	{commit}
	git commit -m ``tresc commita''
	\end{block}
	\begin{block}<5->
	{wysalanie}
	git push origin master
	\end{block}
\end{frame}
\begin{frame}
	\frametitle{Uproszczony schemat pracy}
	\begin{block}<1->
	{pobranie danych z~serwera}
	git fetch
	\end{block}
	\begin{block}<2->
	{mergowanie zmian}
	git merge origin/master
	\end{block}
	\begin{block}<3->
	{praca nad plikiem}
	vim plik\\
	git add plik\\
	git commit -m ``opis''\\
	git push origin
	\end{block}
\end{frame}
\subsection{Trochę ogłady}
\begin{frame}
	\frametitle{Podstawowa konfiguracja}
	\begin{block}<1->
	{przedstawiamy się}
	git config -{}-global user.name ``Jan Kowalski''\\
	git config -{}-global user.email ``moj@email.pl''
	\end{block}
	\begin{block}<2->
	{cenimy sobie komfort pracy}
	git config -{}-global core.editor vim
	\end{block}
\end{frame}
\section{Podstawy II}
\subsection{Branching}
\begin{frame}
	\frametitle{Dwoimy się i~troimy}	
	\begin{block}<1->
	{tworzymy gałąź}
	git branch galaz1
	\end{block}
	\begin{block}<2->
	{przełączamy się}
	git checkout galaz1
	\end{block}
	\begin{block}<3->
	{normalna praca}
	vim plik\\
	git add plik\\
	git commit -m ``opis''
	\end{block}
\end{frame}
\begin{frame}
	\frametitle{Czas na magię\ldots}
	\begin{block}<1->
	{przechodzimy na gałąź główną}
	git checkout master
	\end{block}
	\begin{block}<2->
	{owoc naszych prac\ldots}
	cat plik
	\end{block}
	\begin{block}<3->
	{wracamy na galaz1}
	git checkout galaz1\\
	cat plik
	\end{block}
\end{frame}
\begin{frame}
	\frametitle{Co człowiek rozłączył, git niech złączy\ldots}
	\begin{block}<1->
	{przechodzimy na gałąź do której chcemy dołączyć}
	git checkout master
	\end{block}
	\begin{block}<2->
	{gałęzie\ldots łączcie się}
	git merge galaz1
	\end{block}
	\begin{block}<3->
	{i~oto efekt}
	cat plik
	\end{block}
\end{frame}
\begin{frame}
	\frametitle{Mergowanie}
	Fast forward vs no\dywiz fast forward
\end{frame}
\begin{frame}
	\frametitle{Gałęzie zdalne}
	\begin{block}<1->
	{``importowanie'' gałęzi zdalnych}
	git checkout -t origin/galaz
	\end{block}
	\begin{block}<2->
	{wysyłanie zmian do gałęzi zdalnej}
	git push origin galaz:galaz
	\end{block}
	\begin{block}<3->
	{usuwanie gałęzi zdalnej}
	git push origin :galaz
	\end{block}
\end{frame}
\section{Średnio-zaawansowanie}
\begin{frame}
	\frametitle{Tematy}
		\begin{itemize}
		\item git remote
		\item git archive
		\item git rebase
		\item git revert
		\item git stash
		\item git tag
		\item git clean
		\item git log
		\item git blame
		\item git format-patch
		\item git am
		\item git bisect
		\end{itemize}
\end{frame}
\end{document}
