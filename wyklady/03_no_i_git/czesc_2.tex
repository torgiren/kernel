\documentclass[10pt]{beamer}
\usepackage[utf8]{inputenc}
\usepackage{polski}
\author{Marcin TORGiren Fabrykowski}
\title{No i Git!\\część II}
\institute{AGH - University of Science and Technology}
\usetheme{Warsaw}
\begin{document}
\begin{frame}
	\titlepage
\end{frame}
\section{Instalacja}
\subsection{Przygotowanie}
\begin{frame}
	\frametitle{Przygotowań krok pierwszy}
	\begin{block}<1->
	{tworzymy użytkownika}
	sudo useradd -m git\\
	sudo passwd git
	\end{block}
	\begin{block}<2->
	{klucz do repo}	
	scp .ssh/id\_rsa.pub git@localhost:me.pub
	\end{block}
\end{frame}
\subsection{Instalacja}
\begin{frame}
	\frametitle{Instalacja}
	\begin{block}
	{logowanie}
	ssh git@localhost
	\end{block}
	\begin{block}<2->
	{klonujemy serwer}
	git clone git://github.com/sitaramc/gitolite.git
	\end{block}
	\begin{block}<3->
	{instalujemy}
	gitolite/src/gitolite setup me.pub
	\end{block}
\end{frame}
\section{Konfiguracja}
\subsection{Konfiguracja}
\begin{frame}
	\frametitle{Konfiguracja}	
	\begin{block}<1->
	{repozytorium konfiguracyjne}
	git clone git@localhost:gitolite-admin.git
	\end{block}
	\begin{block}<2->
	{klucze}
	gitolite-admin/keydir
	\end{block}
	\begin{block}<3->
	{konfiguracja}
	conf/gitolite.conf
	\end{block}
\end{frame}
\section{gitolite.conf}
\subsection{Prawa dostępu}
\begin{frame}
	\frametitle{gitolite.conf}
	\begin{block}<1->
	{skladnia}
	każdy widzi :P
	\end{block}
	\begin{block}<2->
	{prawa dostępu}
	-\\
	R\\
	RW\\
	RW+
	\end{block}
\end{frame}
\subsection{Prawa dla gałęzi}
\begin{frame}
	\frametitle{HAHA, jesteśmy panami repo}
	\begin{block}<1->
	{odmówimy dostępu do repo}
	repo prywatne\\
	\ \ \ \ - = badguy\\
	\ \ \ \ RW+ = me
	\end{block}
	\begin{block}<2->
	{prawa do galezi}\
	R master = user1 user2 user3\\
	- master = user1 user2 user3\\
	RW+ = admin1 admin2 admin2\\
	RW+ testing = user1 user2 user3 admin1 admin2 admin3
	\end{block}
\end{frame}
\subsection{Grupy}
\begin{frame}
	\frametitle{Grupujemy się}	
	\begin{block}<1->
	{tworzymy grupy}
	@users = user1 user2 user3\\
	@admins = admin1 admin2 admin3\\
	@team1 = @users admin1
	\end{block}
	\begin{block}<2->
	{przykład z poprzedniego slajdu, ale z grupami}
	R master = @users\\
	- master = @users\\
	RW+ = @admins\\
	RW+ testing = @users
	\end{block}
\end{frame}
\subsection{Nazwy specjalne}
\begin{frame}
	\frametitle{Nazwy specjalne}
	\begin{block}<1->
	{dostęp do gałęzi przywatnych}
	RW+ personal/USER/ = @users
	\end{block}
	\begin{block}<2->
	{dostęp do konkretnych plików}
	R NAME/read\_only\_file = user\\
	- NAME/read\_only\_file = user\\
	RW = user
	\end{block}
\end{frame}
\end{document}
