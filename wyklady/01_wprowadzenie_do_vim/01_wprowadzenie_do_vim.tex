\documentclass{beamer}
\usepackage{polski}
\usepackage[utf8]{inputenc}
\author{Marcin TORGiren Fabrykowski}
\title{Wprowadzenie do VIMa}
\institute{AGH - University of Science and Technology}
\usetheme{Frankfurt}
\begin{document}
\begin{frame}
	\titlepage
\end{frame}
\begin{frame}
	\frametitle{O czym będziemy mówić...}
	\tableofcontents
\end{frame}

\section{Co to jest VIM?}
\begin{frame}
	\frametitle{Co to jest?}
	\begin{itemize}[<+->]
		\item Czym jest VIM?
		\begin{block}{Vim}
			Edytor tekstu
		\end{block}
		\item A co z Officem?
		\begin{block}{... Office}
			Procesor tekstu
		\end{block}
	\end{itemize}
\end{frame}

\section{Podstawy VIM}
\subsection{Wejście i wyjście z VIMa}
\begin{frame}
	\frametitle{Wejście i wyjście z VIMa}
	\begin{itemize}
		\item<1-> Jak włączyć VIMa?
		\begin{block}{Włączenie VIMa}<2->
		vim
		\end{block}
		\item<3-> Jak wyłączyć VIMa?
		\onslide<4->{ ...ale bez generowania losowego ciągu znaków?}
		\begin{block}{Wyłączenie VIMa}<5->
		:q
		\end{block}
	\end{itemize}

\end{frame}
\subsection{Tryby w VIMie}
\begin{frame}
	\frametitle{Tryby w VIMie}
	\begin{itemize}[<+->]
		\item A co jeśli byśmy chcieli wpisać \textit{:q} do pliku?
		\item Dwa tryby pracy:
			\begin{enumerate}
			\item normalny
			\item wstawiania
			\end{enumerate}
		\item Polecenia działają w trybie normalnym
		\begin{block}{Przejscie do tryby normalnego}
			ESC
		\end{block}
	\end{itemize}
\end{frame}

\section{Koniec}
\begin{frame}
	Dziękuję za uwagę
\end{frame}
\end{document}
